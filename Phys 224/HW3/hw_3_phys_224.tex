% ===============================================
% Phys 224: Thermal Physics          Autumn 2019
% hw_template_phys_224.tex
% ===============================================
\documentclass{article}
\usepackage{geometry,amsmath,amsthm,amssymb,hyperref,textcomp,gensymb,siunitx,pgfplots,chemfig}
\geometry{
    letterpaper,
    left=1in,
    top=0.75in,
}
\begin{document}

\large
% -----------------------------
%     Content
% -----------------------------
{\Large Pongpak Techagumthorn
\hfill Phys 224: Thermal Physics}
\begin{center}
    {\Large
        % -----------------------------
        %    HW Number
        % -----------------------------
        HW 3
    }
    \end{center}
\vspace{0.1in}
% ----------------------------------------------------
\begin{enumerate}
    \item [2.24]
        \begin{enumerate}
            \item
            \begin{align*}
                \Omega \left( N,\frac{N}{2} \right) &= \frac{N!}{\left( \frac{N}{2} \right)!\left( \frac{N}{2} \right)!} \\
                &\approx \frac{N^N E^{-N} \sqrt{2 \pi N}}{\left( \left( \frac{N}{2} \right)^{(N/2)} E^{-(N/2)} \sqrt{\pi N} \right)\left( \left( \frac{N}{2} \right)^{(N/2)} E^{-(N/2)} \sqrt{\pi N} \right)} \\
                &\approx \frac{N^N E^{-N} \sqrt{2 \pi N}}{\left( \frac{N}{2} \right)^N E^{-N} \pi N} \\
                &\approx \frac{2^N \sqrt{2 \pi N}}{\pi N} \\
                &\approx \frac{2^{N + 1}}{\sqrt{2 \pi N}} \\
                &\approx \frac{2^{N + \frac{1}{2}}}{\sqrt{2 \pi}} \\
                \Omega_{max} &\approx 2^N
            \end{align*}
            % ----------------------------------------------------
            \vspace{0.05in}
            % ----------------------------------------------------
            \item 
            \begin{align*}
                x &\equiv N \uparrow - \frac{N}{2} \\
                N \uparrow &= \frac{N}{2} + 2
            \end{align*}
            \begin{align*}
                N &= N \uparrow + N \downarrow \\
                N &= \frac{N}{2} + x + N \downarrow \\
                \frac{N}{2} &= x + N \downarrow \\
                N \downarrow &= \frac{N}{2} - x
            \end{align*}
            \begin{equation*}
                \Omega = \binom{N \downarrow + N \uparrow + 1}{N \downarrow} = \binom{\frac{N}{2} + x + \frac{N}{2} - x + 1}{\frac{N}{2} - x}
            \end{equation*}
            \begin{align*}
                \Omega &\approx \binom{N}{\frac{N}{2} - x} = \frac{N!}{\left( \frac{N}{2} - x \right)!\left( \frac{N}{2} - x \right)!} \\
                &\approx \frac{N^N e^{-N} \sqrt{2 \pi N}}{\left( \left( \frac{N}{2} - x \right)^{(\frac{N}{2} - x)} e^{-(\frac{N}{2} - x)} \sqrt{2 \pi (\frac{N}{2} - x)} \right) \left( \left( \frac{N}{2} + x \right)^{(\frac{N}{2} + x)} e^{-(\frac{N}{2} + x)} \sqrt{2 \pi (\frac{N}{2} + x)} \right)} \\
                &\approx \frac{N^N \sqrt{2 \pi N}}{\left( \left( \frac{N}{2} - x \right)^{(\frac{N}{2} - x)} \sqrt{2 \pi (\frac{N}{2} - x)} \right) \left( \left( \frac{N}{2} + x \right)^{(\frac{N}{2} + x)} \sqrt{2 \pi (\frac{N}{2} + x)} \right)} \\
                &\approx \frac{N^N \sqrt{2 \pi N}}{2 \pi \sqrt{(\frac{N}{2})^2 - x^2} \left( \frac{N}{2} - x \right)^{(\frac{n}{2} - x)} \left( \frac{N}{2} + x \right)^{(\frac{N}{2} + x)}} \\
                &\approx \frac{N^N \sqrt{2 \pi N}}{2 \pi \sqrt{(\frac{N}{2})^2 - x^2} \left( \frac{N}{2} - x \right)^{\frac{N}{2}} \left( \frac{N}{2} - x \right)^{-x} \left( \frac{N}{2} + x \right)^{\frac{N}{2}} \left( \frac{N}{2} + x \right)^{x}} \\
                &\approx \frac{N^N \sqrt{2 \pi N}}{2 \pi \sqrt{(\frac{N}{2})^2 - x^2} \left( \left( \frac{N}{2} \right)^2 - x^2 \right)^{\frac{N}{2}} \left( \frac{N}{2} - x \right)^{-x} \left( \frac{N}{2} + x \right)^{x}} \\
                &\approx \frac{N^N \sqrt{N}}{\sqrt{2 \pi (\frac{N}{2})^2 - x^2} \left( \left( \frac{N}{2} \right)^2 - x^2 \right)^{\frac{N}{2}} \left( \frac{N}{2} - x \right)^{-x} \left( \frac{N}{2} + x \right)^{x}} \\
                \Omega &\approx \frac{N^N}{\left( \left( \frac{N}{2} \right)^2 - x^2 \right)^{\frac{N}{2}} \left( \frac{N}{2} - x \right)^{-x} \left( \frac{N}{2} + x \right)^{x}} \\
                \ln(\Omega) &\approx \ln\left( \frac{N^N}{\left( \left( \frac{N}{2} \right)^2 - x^2 \right)^{\frac{N}{2}} \left( \frac{N}{2} - x \right)^{-x} \left( \frac{N}{2} + x \right)^{x}} \right) \\
                &\approx \ln(N^N) - \ln\left( \left( \left( \frac{N}{2} \right)^2 - x^2 \right)^{N/2} \right) - \ln\left( \left( \frac{N}{2} - x \right)^{-x} \right) - \ln\left( \left( \frac{N}{2} + x \right)^x \right) \\
                &\approx N\ln(N) - \frac{N}{2}\ln\left( \left( \frac{N}{2} \right)^2 - x^2 \right) + x\ln\left( \frac{N}{2} - x \right) - x\ln\left( \frac{N}{2} + x \right)
            \end{align*}
            \begin{enumerate}
                \item 
                \begin{align*}
                    \frac{N}{2}\ln\left( \left( \frac{N}{2} \right)^2 - x^2 \right) &= \frac{N}{2}\ln\left( \left( \frac{N}{2} \right)^2 \left( 1 - \left( \frac{2x}{N} \right)^2 \right) \right) \\
                    &= \frac{N}{2}\ln\left( \left( \frac{N}{2} \right)^2 \right) + \frac{N}{2}\ln\left( 1 - \left( \frac{2x}{N} \right)^2 \right) \\
                    &= N\ln\left( \frac{N}{2} \right) - \frac{N}{2}\left( \frac{2x}{N} \right)^2 \\
                    &= N\ln(N) - N\ln(2) - \frac{2x^2}{N}
                \end{align*}
                % ----------------------------------------------------
                \vspace{0.05in}
                % ----------------------------------------------------
                \item
                \begin{align*}
                    x\ln\left( \frac{N}{2} - x \right) &= x\ln\left( \frac{N}{2} \left( 1 - \frac{2x}{N} \right) \right) \\
                    &= x\ln\left( \frac{N}{2} \right) + x\ln\left( 1 - \frac{2x}{N} \right) \\
                    &= x\ln\left( \frac{N}{2} \right) - x\left( \frac{2x}{N} \right) \\
                    &= x\ln\left( \frac{N}{2} \right) - \frac{2x^2}{N}
                \end{align*}
                % ----------------------------------------------------
                \vspace{0.05in}
                % ----------------------------------------------------
                \item
                \begin{align*}
                    x\ln\left( \frac{N}{2} + x \right) &= x\ln\left( \frac{N}{2} \left( 1 + \frac{2x}{N} \right) \right) \\
                    &= x\ln\left( \frac{N}{2} \right) + x\ln\left( 1 + \frac{2x}{N} \right) \\
                    &= x\ln\left( \frac{N}{2} \right) + x\left( \frac{2x}{N} \right) \\
                    &= x\ln\left( \frac{N}{2} \right) + \frac{2x^2}{N}
                \end{align*}
            \end{enumerate}
            % ----------------------------------------------------
            \vspace{0.025in}
            % ----------------------------------------------------
            \begin{align*}
                \ln(\Omega) &\approx N\ln(N) - \left( N\ln(N) - N\ln(2) - \frac{2x^2}{N} \right) + \left( x\ln\left( \frac{N}{2} \right) - \frac{2x^2}{N} \right) - \left( x\ln\left( \frac{N}{2} \right) + \frac{2x^2}{N} \right) \\
                \ln(\Omega) &\approx N\ln(2) - \frac{2x^2}{N} \\
                \Omega &\approx 2^N e^{-\frac{2x^2}{N}} \\
                \Omega &\approx 2^N \quad (\Omega_{max}, x = 0)
            \end{align*}
            % ----------------------------------------------------
            \vspace{0.05in}
            % ----------------------------------------------------
            \item
            Width measured when value falls to \(\frac{1}{e}\) of max value.
            \begin{align*}
                \frac{\Omega_{max}}{e} &= \Omega_{max} e^{-\frac{2x^2}{N}} \\
                e^{-1} &= e^{-\frac{2x^2}{N}} \\
                1 &= \frac{2x^2}{N} \\
                x^2 &= \frac{N}{2} \\
                x &= \sqrt{\frac{N}{2}}
            \end{align*}
            Total width \(w = 2x = 2\sqrt{\frac{N}{2}} = \sqrt{2N}\).
            % ----------------------------------------------------
            \vspace{0.05in}
            % ----------------------------------------------------
            \item
            \begin{equation*}
                P = \frac{\Omega}{\Omega_{max}} = \frac{\Omega_{max} e^{-\frac{2x^2}{N}}}{\Omega_{max}} = e^{-\frac{2x^2}{N}}
            \end{equation*}
            \begin{equation*}
                P_{501000} = e^{-\frac{2(1000)^2}{1000000}} = 0.135
            \end{equation*}
            501000 heads is probable
            \begin{equation*}
                P_{510000} = e^{-\frac{2(10000)^2}{1000000}} = 1.38 \times 10^{-87}
            \end{equation*}
            510000 heads is improbable
        \end{enumerate}
    % ----------------------------------------------------
    \vspace{0.1in}
    % ----------------------------------------------------
    \item [2.25]
        \begin{enumerate}
            \item
                N = Total Steps, n = Steps away from starting position
                \[\Omega \approx \frac{N!}{n!(N - n)!}\]
                \[\Omega_{max} \approx 2^N\]
                \[P = \frac{\Omega}{\Omega_{max}} = \frac{N!}{n!(N-n)!} \frac{1}{2^N}\]
                Highest probability state is when \(n = 0\), so ending up at the same place.
                It is also probable that you will end up in a spot that is within the width of the peak, so \(\pm \sqrt{\frac{N}{2}}\) steps away from the starting position.
            \item
                max \(\Delta x = \pm \sqrt{\frac{10000}{2}} \approx \pm 71\). \\
                It is likely that you will end up within \(\pm\) 71 steps from the starting point.
            \item
                Mean free path:
                \[\ell = \frac{1}{4 \pi r^2} \frac{V}{N_{mol}}\]
                \[\overline{v} = \sqrt{\frac{3kT}{m}}\]
                Typical values for air at STP: \(\ell = 1.5 \times 10^{-7}m\) and \(\overline{v} = 500 m/s\).
                \[N = \frac{\overline{v} t}{\ell}\]
                \begin{align*}
                    \Delta x &\approx x_{width} \ell = \sqrt{\frac{N}{2}} \ell = \sqrt{\frac{\overline{v} t}{2 l}} \ell \\
                    &\approx \sqrt{\frac{\overline{v} t \ell}{2}} \\
                    \Delta x &\approx \sqrt{\frac{(500)(1)(1.5 \times 10^{-7})}{2}} = 0.0061 m
                \end{align*}
                \(\Delta x\) depends on \(\sqrt{t}\) so for an increasing \(\Delta x\) the time it takes will grow at an exponential rate. \\
                \(\Delta x\) also has a dependence on \(\sqrt[4]{T}\) so for increases in temperature \(\Delta x\) does not appreciably change.
        \end{enumerate}
    % ----------------------------------------------------
    \vspace{0.1in}
    % ----------------------------------------------------
    \item [2.29]
        \[\Omega_{most} = 6.87 \times 10^{114}\]
        \[\Omega_{least} = 2.77 \times 10^{81}\]
        \[S_{most} = \ln(\Omega_{most}) = \ln(6.87 \times 10^{114}) = 264.4\]
        \[S_{least} = \ln(\Omega_{least}) = \ln(2.77 \times 10^{81}) = 187.5\]
        \[\Omega_{total} = 9.26 \times 10^{115}\]
        \[S_{total} = \ln(\Omega_{total}) = \ln(9.26 \times 10^{115}) = 267\]
    % ----------------------------------------------------
    \vspace{0.1in}
    % ----------------------------------------------------    
    \item [2.34]
        \begin{align*}
            S &= k\ln(\Omega) \\
            &= Nk\left[ \ln\left( \frac{V}{N} \left( \frac{4n \pi U}{3Nh^3} \right)^{\frac{3}{2}} \right) + \frac{5}{2} \right] \quad DoF = 3
        \end{align*}
        \begin{align*}
            Q &= \Delta U + W \quad \Delta U = 0 \\
            &= W = \int_{V_i}^{V_f} P dV \quad P = \frac{NkT}{V} \\
            &= NkT \int_{V_i}^{V_f} \frac{dV}{V} = NkT \ln\left( \frac{V_f}{V_i} \right)
        \end{align*}
        \begin{align*}
            S &= Nk\left[ \ln\left( \frac{V}{N} \left( \frac{4n \pi U}{3Nh^3} \right)^{\frac{3}{2}} + \frac{5}{2} \right) \right] \\
            &= Nk\left[ \ln(V) \ln\left( \frac{1}{N} \left( \frac{4n \pi U}{3Nh^3} \right)^{\frac{3}{2}} \right) + \frac{5}{2} \right]
        \end{align*}
        \(\ln\left( \frac{1}{N} \left( \frac{4n \pi U}{3Nh^3} \right)^{\frac{3}{2}} \right) + \frac{5}{2}\) is constant
        \begin{align*}
            \Delta S &= Nk\ln(V_f) - Nk\ln(V_i) = Nk\ln\left( \frac{V_f}{V_i} \right) \\
            &= Nk\ln\left( \frac{V_f}{V_i} \right) \frac{T}{T} = \frac{NkT\ln\left( \frac{V_f}{V_i} \right)}{T} \\
            \Delta S &= \frac{Q}{T}
        \end{align*}
    % ----------------------------------------------------
    \vspace{0.1in}
    % ----------------------------------------------------
    \item [2.36]
        \[S \approx Nk\]
        \[N_{book} = \frac{1000}{12} \times 6.022 \times 10^{22} = 5.02 \times 10^{25}\]
        \[S_{book} = 5.02 \times 10^{25} \times 1.38 \times 10^{-23} = 693 JK^{-1}\]
        \[N_{moose} = \frac{400 \times 10^3}{18} \times 6.022 \times 10^{22} = 1.34 \times 10^{28}\]
        \[S_{moose} = 1.34 \times 10^{28} \times 1.38 \times 10^{-23} = 1.84 \times 10^5 JK^{-1}\]
        \[N_{sun} = \frac{2 \times 10^{33}}{12} \times 6.022 \times 10^{22} = 1.2 \times 10^{57}\]
        \[S_{sun} = 1.2 \times 10^{57} \times 1.38 \times 10^{-23} = 1.66 \times 10^{34} JK^{-1}\]
    % ----------------------------------------------------
    \vspace{0.1in}
    % ----------------------------------------------------
    \item [3.1]
        \[\frac{1}{T} = \frac{\partial S}{\partial U} \]
        \[T = \frac{\partial U}{\partial S} = \frac{\Delta U}{\Delta S} = \frac{\epsilon \Delta q}{\Delta S}\]
        \[\epsilon = 0.1 eV = 1.6 \times 10^{-20} J\]
        \begin{enumerate}
            \item \(q_a = 1\)
            \begin{align*}
                T_A &= \frac{\epsilon (2 - 0)}{S_A(2) - S_A(0)} = \frac{2 \epsilon}{k(10.7 - 0)} = 0.187 \frac{\epsilon}{k} \\
                &= 0.187 \left( \frac{1.6 \times 10^{-20}}{1.38 \times 10^{-23}} \right) = 216.7 K
            \end{align*}
            \begin{align*}
                T_B &= \frac{\epsilon (100 - 98)}{S_B(100) - S_B(98)} = \frac{2 \epsilon}{k(187.53 - 185.33)} = 0.909 \frac{\epsilon}{k} \\
                &= 0.909 \left( \frac{1.6 \times 10^{-20}}{1.38 \times 10^{-23}} \right) = 1054 K
            \end{align*}
            \item \(q_a = 60\)
            \begin{align*}
                T_A &= \frac{\epsilon (61 - 59)}{S_A(61) - S_A(59)} = \frac{2 \epsilon}{k(160.9 - 157.35)} = 0.56 \frac{\epsilon}{k} \\
                &= 0.56 \left( \frac{1.6 \times 10^{-20}}{1.38 \times 10^{-23}} \right) = 649 K
            \end{align*}
            \begin{align*}
                T_B &= \frac{\epsilon (41 - 39)}{S_B(41) - S_B(39)} = \frac{2 \epsilon}{k(107.0 - 103.5)} = 0.56 \frac{\epsilon}{k} \\
                &= 0.56 \left( \frac{1.6 \times 10^{-20}}{1.38 \times 10^{-23}} \right) = 649 K
            \end{align*}
            \[T_A = T_B\]
        \end{enumerate}
    % ----------------------------------------------------
    \vspace{0.1in}
    % ----------------------------------------------------
    \item [3.2]
        \[\frac{1}{T} = \frac{\partial S}{\partial U} \Rightarrow T = \frac{\partial U}{\partial S}\]
        \[T_A = \frac{\partial U_A}{\partial S_A} \quad T_B = \frac{\partial U_B}{\partial S_B} \quad T_C = \frac{\partial U_C}{\partial S_C}\]
        \begin{align*}
            T_A &= T_B \\
            \frac{\partial U_A}{\partial S_A} &= \frac{\partial U_B}{\partial S_B} \\
            T_B &= T_C \\
            \frac{\partial U_B}{\partial S_B} &= \frac{\partial U_C}{\partial S_C} \Rightarrow \frac{\partial U_A}{\partial S_A} = \frac{\partial U_C}{\partial S_C} \Rightarrow T_A = T_C
        \end{align*}
    % ----------------------------------------------------
    \vspace{0.1in}
    % ----------------------------------------------------
    \item [3.3]
        The two objects will exchange thermal energy until the slopes of their entropy vs energy graphs are equal.
    % ----------------------------------------------------
    \vspace{0.1in}
    % ----------------------------------------------------
    \item [3.5]
        \[\frac{1}{T} = \frac{\partial S}{\partial U} \quad \Omega = \left( \frac{Ne}{q} \right)^q\]
        \[U = q \epsilon \Rightarrow q = \frac{U}{\epsilon}\]
        \[\omega = \left( \frac{Ne \epsilon}{U} \right)^{\frac{U}{\epsilon}}\]
        \begin{align*}
            S = k \ln(\Omega) &= k\ln\left[ \left( \frac{Ne \epsilon}{U} \right)^{\frac{U}{\epsilon}} \right] \\
            &= \frac{Uk}{\epsilon} \ln\left( \frac{Ne \epsilon}{U} \right) \\
            &= \frac{Uk}{\epsilon}(\ln(N \epsilon) + \ln(e) - \ln(U)) \\
            \frac{1}{T} = \frac{\partial S}{\partial U} &= \frac{\partial}{\partial U}\left[ \frac{Uk}{\epsilon}(\ln(N \epsilon) + 1 - \ln(U)) \right] \\
            &= \frac{\partial}{\partial U}\left[ \frac{Uk}{\epsilon} \ln(N \epsilon) + \frac{Uk}{\epsilon} - \frac{Uk}{\epsilon} \ln(U) \right] \\
            &= \frac{k}{\epsilon} \ln(N \epsilon) + \frac{k}{\epsilon} - \left( \frac{k}{\epsilon} \ln(U) + \frac{Uk}{U \epsilon} \right) \\
            \frac{1}{T} &= \frac{k}{\epsilon} (\ln(N \epsilon) - \ln(U)) \\
            \frac{\epsilon}{kT} &= \ln(N \epsilon) - \ln(U) \\
            e^{\ln(U)} &= e^{\left( \ln(N \epsilon) - \frac{\epsilon}{kT} \right)} \\
            U &= N \epsilon e^{-\frac{\epsilon}{kT}}
        \end{align*}
    % ----------------------------------------------------
    \vspace{0.1in}
    % ----------------------------------------------------
    \item [3.10]
        \begin{enumerate}
            \item
                Heat of formation for water \(L_{H_20(\ell)} = 334 Jg^{-1}\)
                \[Q = mL = (30)(334) = 10020 J\]
                \[\Delta S_A = \frac{Q}{T} = \frac{10020}{273} = 36.7 JK^{-1}\]
            \item
                \[c_{H_2O(\ell)} = 4.181 Jg^{-1}K^{-1}\]
                \[\Delta S_B = C_V \int_{T_i}^{T_f} \frac{1}{T} dT\]
                \[C_V = mc\]
                \begin{align*}
                    \Delta S_B &= mc \int_{T_i}^{T_f} \frac{1}{T} dT \\
                    &= mc \ln\left( \frac{T_f}{T_i} \right) \\
                    &= (30)(4.181) \ln\left( \frac{298}{273} = 11 JK^{-1} \right)
                \end{align*}
                \[\Delta S_{water} = \Delta S_A + \Delta S_B = 36.7 + 11 = 47.7 JK^{-1}\]
            \item
                \begin{align*}
                    Q &= Q_{ice} + Q_{water} = Q_{ice} + mc \Delta T \\
                    &= 10020 + (30)(4.181)(25) = 13156 J
                \end{align*}
                \[\Delta S_{room} = \frac{-13156}{298} = -44.7 JK^{-1}\]
            \item
                \[\Delta S_{univ} = \Delta S_{room} + \Delta S_{water} = -44.1 + 47.7 = 3.6 JK^{-1}\]
                A positive entropy is expected since the reaction occurs spontaneously
        \end{enumerate}
    % ----------------------------------------------------
    \vspace{0.1in}
    % ----------------------------------------------------
    \item [3.11]
        \[T_1 = 55C = 328K \quad T_2 = 10C = 283K \quad V_1 = 50L \quad V_2 = 25 L\]
        \begin{align*}
            T_f &= \frac{T_1 V_1 + T_2 V_2}{V_1 + V_1} \\
            &= \frac{(328)(50) + (283)(25)}{50 + 25} = 313 K
        \end{align*}
        \[\Delta S = C_V \int_{T_i}^{T_f} \frac{1}{T} dT = C_v \ln \left( \frac{T_f}{T_i} \right) = mc \ln\left( \frac{T_f}{T_i} \right)\]
        \[\Delta S_{hot} = (50000)(4.181) \ln\left(\frac{313}{328}\right) = -9786 JK^{-1}\]
        \[\Delta S_{cold} = (25000)(4.181) \ln\left(\frac{313}{283}\right) = 10532 JK^{-1}\]
        \[\Delta S_{univ} = \Delta S_{hot} + \Delta S_{cold} = -9786 + 10532 = 746 JK^{-1}\]
    % ----------------------------------------------------
    \vspace{0.1in}
    % ----------------------------------------------------
    \item [3.13]
        \begin{enumerate}
            \item
                \[\Delta S = \frac{Q}{T} \quad Q = F_S T\]
                \[F_s = 1000 Wm^{-2} \quad t = 365 \times 8 \times 60 \times 60 = 10512000 s\]
                \[Q = (1000)(10512000) = 1.0512 \times 10^{10}\]
                \[\Delta S_{\bigoplus} = \frac{1.0512 \times 10^{10}}{300} = 3.5 \times 10^7 JK^{-1}\]
                \[\Delta S_{\bigodot} = \frac{-1.0512 \times 10^{10}}{6000} = -1.75 \times 10^6 JK^{-1}\]
                \[\Delta S_{univ} = \Delta S_{\bigoplus} + \Delta S_{\bigodot} = 3.5 \times 10^7 - 1.75 \times 10^6 = 3.325 JK^{-1}\]
            \item
                In order for the grass to grow it needs sunlight. The entropy that is gained in the 1 square meter of grass is many magnitudes higher than the entropy lost in the formation of complex hydrocarbons.
        \end{enumerate}
    % ----------------------------------------------------
    \vspace{0.1in}
    % ----------------------------------------------------
    \item [3.14]
        \[C_V = aT + bT^3 \quad a= 0.00135 \quad b = 2.48 \times 10^{-5}\]
        \begin{align*}
            S(T_f) - S(T_i) &= \int_{T_i}^{T_f} \frac{C_V(T)}{T} dT \\
            S(T_f) - S(0) =& \int_{0}^{T_f} (a + bT^2) dT \\
            S(T_f) &= \left. \left( aT + \frac{bT^3}{3} \right) \right|_{T = 0}^{T = T_f} \\
            S(T_f) &= aT_f + \frac{bT_f^2}{3}
        \end{align*}
        \begin{align*}
            S(1) &= (0.00135)(1) + \frac{2.48 \times 10^{-5}}{3} (1)^3 = 1.36 \times 10^{-3} JK^{-1} \\
            \frac{S(1)}{k} &= \frac{1.36 \times 10^{-3}}{1.38 \times 10^{-23}} = 9.86 \times 10^{19}
        \end{align*}
        \begin{align*}
            S(10) &= (0.00135)(10) + \frac{2.48 \times 10^{-5}}{3} (10)^3 = 2.18 \times 10^{-2} JK^{-1} \\
            \frac{S(10)}{k} &= \frac{2.18 \times 10^{-2}}{1.38 \times 10^{-23}} = 1.58 \times 10^{21}
        \end{align*}
% ---------------------------------------------------
%     End
% ---------------------------------------------------
\end{enumerate}
\end{document}