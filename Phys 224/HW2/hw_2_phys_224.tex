% ===============================================
% Phys 224: Thermal Physics          Autumn 2019
% hw_template_phys_224.tex
% ===============================================
\documentclass{article}
\usepackage{geometry,amsmath,amsthm,amssymb,hyperref,textcomp,gensymb,siunitx,pgfplots,chemfig}
\geometry{
    letterpaper,
    left=1in,
    top=0.75in,
}
\begin{document}

\large
% -----------------------------
%     Content
% -----------------------------
{\Large Pongpak Techagumthorn
\hfill Phys 224: Thermal Physics}
\begin{center}
    {\Large
        % -----------------------------
        %    HW Number
        % -----------------------------
        HW 2
    }
    \end{center}
\vspace{0.1in}
% ----------------------------------------------------
\begin{enumerate}
    \item [1.22]
    \begin{enumerate}
        \item
            \begin{equation*}
                \overline{P} = \frac{\overline{F_{x,piston}}}{A} = - \frac{\overline{F_{x,molecule}}}{A} = - \frac{m \left( \frac{\overline{\Delta v_x}}{\Delta t} \right)}{A}
            \end{equation*}
            \begin{equation*}
                \overline{P} = \frac{P}{N}
            \end{equation*}
            \begin{equation*}
                P = - \frac{Nm \left( \frac{\overline{\Delta v_x}}{\Delta t} \right)}{A} = - \frac{Nm \overline{v_x}}{A \Delta t}
            \end{equation*}
            \begin{align*}
                PA \Delta t &= -Nm \overline{\Delta v_x} \\
                \frac{PA \Delta t}{m \overline{\Delta v_x}} &= -N
            \end{align*}
            \begin{equation*}
                \overline{\Delta v_x} = \overline{v_f} - \overline{v_i} = - \overline{v_x} - \overline{v_x} = 2 \overline{v_x}
            \end{equation*}
            \begin{align*}
                \frac{PA \Delta t}{m \left(-2 \overline{v_x} \right)} &= -N \\
                \frac{PA \Delta t}{2m \overline{v_x}} &= N
            \end{align*}
        % ------------------------------------------
        \vspace{0.05in}
        % ------------------------------------------
        \item
            \begin{align*}
                \frac{PA \Delta t}{2m \overline{v_x}} &= N \\
                \frac{P}{N} &= \frac{2m \overline{v_x}}{A \Delta t}
            \end{align*}
            \begin{equation*}
                \Delta t = \frac{2L}{\overline{v_x}}
            \end{equation*}
            \begin{align*}
                \frac{P}{N} &= \frac{2m \overline{v_x}}{A \frac{2L}{\overline{v_x}}} \\
                &= \frac{m \overline{{v_x}^2}}{V} \\
                \frac{PV}{N} &= m \overline{{v_x}^2} \\
                kT = \frac{PV}{N} &= m \overline{{v_x}^2} \\
                kT &= m \overline{{v_x}^2} \\
                \frac{kT}{m} &= \overline{{v_x}^2} \\
                \sqrt{\frac{kT}{m}} &= \sqrt{\overline{{v_x}^2}} = \left(\overline{{v_x}^2} \right)^{\frac{1}{2}}
            \end{align*}
        % ------------------------------------------
        \vspace{0.05in}
        % ------------------------------------------
        \item
            \begin{align*}
                \frac{dN}{dt} &= - \frac{AN}{2V} \sqrt{\frac{kT}{m}} \\
                \frac{dN}{N} &= - \frac{A}{2V} \sqrt{\frac{kT}{m}} dt \\
                \int \frac{1}{N} dN &= - \frac{A}{2V} \sqrt{\frac{kT}{m}} \int dt \\
                \ln (N) &= - \frac{A}{2V} \sqrt{\frac{kT}{m}} t + C \\
                e^{ln (N)} &= e^C e^{\left(- \frac{A}{2V} \sqrt{\frac{kT}{m}} t \right)} \\
                N &= e^C e^{\left(- \frac{A}{2V} \sqrt{\frac{kT}{m}} t \right)}
            \end{align*}
            \begin{equation*}
                \frac{1}{\tau} = \frac{A}{2V} \sqrt{\frac{kT}{m}}
            \end{equation*}
            \begin{equation*}
                N = N_0 e^{- \frac{t}{\tau}}
            \end{equation*}
        % ------------------------------------------
        \vspace{0.05in}
        % ------------------------------------------
        \item
            \begin{equation*}
                \tau = \frac{2V}{A} \sqrt{\frac{m}{kT}}
            \end{equation*}
            Room Temp = 298\si{\degree\kelvin}, V = 1\si{\liter} = 0.001\si{\cubic\metre} = \SI{1e-3}{\cubic\metre}, A = 1\si{\square\milli\metre} = \SI{1e-6}{\square\metre} \\
            Air = 78\% \(N_2\) + 21\% \(O_2\) + 1\% \(Ar\) \\
            \(0.78 \times 2 \times 14.00674 + 0.21 \times 2 \times 15.9994 + 0.01 \times 39.948 = 28.97\)\si{\gram\per\mole} \\
            \(m_{air} = 28.97 \times 10^{-3}/N_a = 4.81 \times 10^{-26}\)\si{\kilo\gram}
            \begin{equation*}
                \tau = \frac{2 \left(1 \times 10^{-3} \right)}{1 \times 10^{-6}} \sqrt{\frac{4.81 \times 10^{-26}}{\left(1.38 \times 10^{-23} \right)(298)}} = 6.32
            \end{equation*}
            \begin{center}
            \(\tau = 6.32\)\si{\second}
            \end{center}
        % ------------------------------------------
        \vspace{0.05in}
        % ------------------------------------------
        \item
            Lunar Command Module (LCM) Volume = 6.2\si{\cubic\metre} \\
            LCM Hatch Size = 0.74\si{\metre} \(\times\) 0.86\si{\metre} = 0.63\si{\square\metre} \\
            LCM Air = 100\% \(O_2\) \\
            \(m_{O_2} = 2 \times 15.9994 / N_a = 5.31 \times 10^{-23}\)\si{\gram} \(= 5.31 \times 10^{-26}\)\si{\kilo\gram}
            \begin{equation*}
                \tau = \frac{6.2}{6.3} \sqrt{\frac{5.31 \times 10^{-26}}{\left(1.38 \times 10^{-23} \right)(298)}} = 0.035
            \end{equation*}
            \begin{center}
                \(\tau = 0.035\)\si{\second}
            \end{center}
    \end{enumerate}
    % -----------------------------------------------
    \vspace{0.1in}
    % -----------------------------------------------
    \item [1.24]
        Degree of freedom \(f = 6\), Temperature = 298\si{\degree\kelvin}
        \begin{equation*}
            U_{therm} = \frac{f}{2} NkT
        \end{equation*}
        \begin{equation*}
            N_{Pb} = \frac{1}{207.2} \times 6.022 \times 10^{23} = 2.906 \times 10^{21}
        \end{equation*}
        \begin{align*}
            U_{therm} &= \frac{f}{2} NkT \\
            &= \left(\frac{6}{2} \right) \left(2.906 \times 10^{21} \right) \left(1.381 \times 10^{-23} \right) (298) \\
            &= 35.878 \si{\joule}
        \end{align*}
    % -----------------------------------------------
    \vspace{0.1in}
    % -----------------------------------------------
    \item [1.27]
    \begin{enumerate}
        \item
            Exothermic chemical reactions such as a thermite reaction.
        \item
            Phase transitions and latent heat. 
    \end{enumerate}
    % -----------------------------------------------
    \vspace{0.1in}
    % -----------------------------------------------
    \item [1.31]
    \begin{enumerate}
        \item 
            Graph: \\
            \begin{tikzpicture}
                \begin{axis}
                    [
                        title={Pressure vs. Volume},
                        xlabel={Pressure (Atm)},
                        ylabel={Volume (Liter)},
                        xmin=0, xmax=4,
                        ymin=0, ymax=4,
                        xtick={0,1,2,3},
                        ytick={0,1,2,3},
                    ]
                    \addplot[
                        color=blue,
                        mark=square,
                    ]
                    coordinates {
                        (1,1)(3,3)
                    };
                \end{axis}
            \end{tikzpicture}
        % ------------------------------------------
        \vspace{0.05in}
        % ------------------------------------------
        \item
            \begin{equation*}
                W = - \int^{V_f}_{v_i}{P(V)dV}
            \end{equation*}
            \begin{equation*}
                P(V) = \mathrm{\frac{2atm}{2L}} V = \frac{101.3 \times 10^{3}\si{\pascal}}{0.001\si{\cubic\meter}} V = 101.3 \times 10^{6} V
            \end{equation*}
            \begin{align*}
                W &= -\int^{3 \times 10^{-3}}_{1 \times 10^{-3}} {101.3 \times 10^{6} VdV} \\
                &= -101.3 \times 10^{6} \int^{0.003}_{0.001} {VdV} \\
                &= -101.3 \times 10^{6} \left[\frac{V^2}{2} \right]^{0.003}_{0.001} \\
                &= -101.3 \times 10^{6} \left[\frac{0.003^2}{2} - \frac{0.001^2}{2} \right] \\
                &= -405.2\si{\joule}
            \end{align*}
        % ------------------------------------------
        \vspace{0.05in}
        % ------------------------------------------
        \item
            \begin{equation*}
                \Delta U = \frac{f}{2}NkT = \frac{f}{2}PV
            \end{equation*}
            \begin{align*}
               \Delta U &= \frac{f}{2} \Delta (PV) \\
               &= \frac{f}{2} \left(P_f V_f - P_i V_i\right)
            \end{align*}
            Degree of freedom \(f = 3\) for Helium at room temperature
            \begin{align*}
                \Delta U &= \frac{3}{2} \left( \left(303.9 \times 10^{3} \right)(0.003) - \left(101.3 \times 10^{3} \right)(0.001) \right) \\
                &= \frac{3}{2} (810.4) \\
                \Delta U &= 1215.6\si{\joule} 
            \end{align*}
        % ------------------------------------------
        \vspace{0.05in}
        % ------------------------------------------
        \item
            \begin{equation*}
                Q = \Delta U - W = 1215.6 - -405.2 = 1621\si{\joule}
            \end{equation*}
        % ------------------------------------------
        \vspace{0.05in}
        % ------------------------------------------
        \item
            Add a lot of heat in a short amount of time, such as an explosion.
    \end{enumerate}
    % -----------------------------------------------
    \vspace{0.1in}
    % -----------------------------------------------
    \item [1.34]
    \begin{enumerate}
        \item
            \begin{equation*}
                W = \int_{V_i}^{V_f} P(V) dV \quad U = \frac{f}{2}NkT
            \end{equation*}
            \begin{equation*}
                PV = NkT \quad f=5
            \end{equation*}
            \begin{equation*}
                U = \frac{5}{2}PV
            \end{equation*}
            \begin{equation*}
                \Delta U = \frac{5}{2} \Delta PV = \frac{5}{2} P \Delta V = Q = W
            \end{equation*}
            \begin{equation*}
                Q = \Delta U - W
            \end{equation*}
            \\ \\
            \begin{align*}
                W_A &= - \int_{V_1}^{V_1} (P_2 - P_1) dV \\
                &= -(P_2 - P_1)(V_1 - V_1) \\
                &= 0
            \end{align*}
            \begin{equation*}
                \Delta U_A = \frac{5}{2} (P_2 - P_1)(V_1)
            \end{equation*}
            \begin{align*}
                Q_a &= \frac{5}{2} (P_2 - P_1)(V_1) - 0 \\
                &= \frac{5}{2} (P_2 - P_1)(V_1)
            \end{align*}
            \\ \\
            \begin{align*}
                W_B &= - \int_{V_1}^{V_2} P_2 dV \\
                &=  -P_2(V_2 - V_1)
            \end{align*}
            \begin{equation*}
                \Delta = \frac{5}{2} (P_2)(V_2 - V_1)
            \end{equation*}
            \begin{equation*}
                Q_B = \frac{5}{2} (P_2)(V_2 - V_2) + P_2(V_2 - V_1)
            \end{equation*}
            \\ \\
            \begin{align*}
                W_C &= -\int_{V_2}^{V_2} (P_1 - P_2) dV \\
                &= - (P_1 - P_2)(V_2 - V_2) = 0
            \end{align*}
            \begin{equation*}
                \Delta U_C = \frac{5}{2} (P_1 - P_2)(V_2)
            \end{equation*}
            \begin{align*}
                Q_C &= \frac{5}{2} (P_1 - P_2)(V_2) - 0 \\
                &= \frac{5}{2} (P_1 - P_2)(V_2)
            \end{align*}
            \\ \\
            \begin{align*}
                W_D &= - \int_{V_2}^{V_1} P_1 dV \\
                &= - P_1(V_1 - V_2)
            \end{align*}
            \begin{equation*}
                \Delta U_D = \frac{5}{2}(P_1)(V_1 - V_2)
            \end{equation*}
            \begin{equation*}
                Q_D = \frac{5}{2} (P_1)(V_1 - V_2) + P_1(V_1 - V_2)
            \end{equation*}
        % ------------------------------------------
        \vspace{0.05in}
        % ------------------------------------------
        \item
            \begin{enumerate}
                \item [A]
                    Heat is added to the system while the piston is fixed.
                \item [B]
                    Heat Is being added to the system while the piston is pilled out to maintain constant pressure
                \item [C]
                    Heat is being removed from the system while the piston is fixed
                \item [D]
                    Heat is being removed from the system while the piston is pushed in to maintain pressure
            \end{enumerate}
        % ------------------------------------------
        \vspace{0.05in}
        % ------------------------------------------
        \item
            \begin{align*}
                W_{total} &= W_A + W_B + W_C + W_D \\
                &= 0 + -P_2(V_2 - V_1) + 0 + -P_1(V_1 - V_2) \\
            \end{align*}
            \begin{equation*}
                V_2 - V_1 = \Delta V
            \end{equation*}
            \begin{align*}
                W_{total} &= -P_2 \Delta V + P_1 \Delta V \\
                &= -(P_2 \Delta V - P_1 \Delta V)
            \end{align*}
            Work is not zero because the volume changes and because the work along B is larger than the work along D.
            \\ \\
            \begin{align*}
                Q_{total} &= Q_A + Q_B + Q_C + Q_D \\
                &= \frac{5}{2} (P_2 - P_1)V_1 + \left[ \frac{5}{2} P_2(V_2 - V_1) + P_2(V_2 - V_1) \right] \\
                & \quad + \frac{5}{2} (P_2 - P_1)V_2 + \left[ \frac{5}{2} P_1(V_1 - V_2) + P_1(V_1 - V_2) \right] \\
                &= P_2(V_2 - V_1) + P_1(V_1 - V_2) \\
                &= P_2 \Delta V - P_1 \Delta V
            \end{align*}
            During B and D heat is has to be added or removed to keep T constant. There is also more heat being added during B than removed during D because of greater pressure at B. This leads to a non zero total heat.
            \\ \\
            \begin{align*}
                \Delta U_{total} &= \Delta U_A + \Delta U_B + \Delta U_C + \Delta U_D \\
                &= \frac{5}{2} (P_2 - P_1)(V_1) + \frac{5}{2} (P_2)(V_2 - V_1) + \frac{5}{2} (P_1 - P_2)(V_2) + \frac{5}{2} (P_1)(V_1 - V_2) \\
                &= 0
            \end{align*}
            The final change in internal energy at the end of the cycle since \(P_i\) and \(V_i\) are the same as \(P_f\) and \(V_f\).
    \end{enumerate}
    % -----------------------------------------------
    \vspace{0.1in}
    % -----------------------------------------------
    \item [1.37]
        \begin{equation*}
            VT^{\frac{f}{2}} = constant \quad PV = NkT
        \end{equation*}
        \begin{align*}
            T &= \frac{PV}{Nk} \\
            T^{\frac{f}{2}} &= \left( \frac{PV}{Nk} \right)^{\frac{f}{2}} \\
            VT^{\frac{f}{2}} &= V \left( \frac{PV}{Nk} \right)^{\frac{f}{2}} \\
            &= V^{\left(1 + \frac{f}{2} \right)} P^{\frac{f}{2}} \left( \frac{1}{Nk} \right)^{\frac{f}{2}} \\
            constant &= V^{\left(1 + \frac{f}{2} \right)} P^{\frac{f}{2}} \left( \frac{1}{Nk} \right)^{\frac{f}{2}} \\
            \frac{constant}{\left( \frac{1}{Nk} \right)^{\frac{f}{2}}} &= V^{\frac{2 + f}{2}} P^{\frac{f}{2}}
        \end{align*}
        N, K, and f are constant
        \begin{align*}
            constant &= V^{\frac{2 + f}{2}} P^{\frac{f}{2}} \\
            constant^{\left( \frac{2}{f} \right)} &= V^{\frac{2}{f} \left(\frac{2 + f}{2} \right)} P^{\frac{2}{f}\frac{f}{2}} \\
            constant &= V^{\frac{2 + f}{f}} P \quad \gamma = \frac{2 + f}{f} \\
            constant &= V^\gamma P
        \end{align*}
    % -----------------------------------------------
    \vspace{0.1in}
    % -----------------------------------------------
    \item [1.37]
        Biodiesel:
        \begin{center}
            \chemfig[angle increment=30]{-[1]-[-1]-[1]-[-1]-[1]-[-1]-[1]-[-1]-[1](=[:90]O)-[-1]O-[1]}
        \end{center}
        Degree of freedom \(f = 6\), \(T_i = 300\)\si{\degree\kelvin}
        \begin{align*}
            V_f T_f^{\frac{f}{2}} &= V_i T_i^{\frac{f}{2}} \\
            \frac{1}{20} T_f^3 &= (1) 300^3 \\
            T_f^3 &= (20) 300^3 \\
            T_f &= \sqrt[3]{(20) 300^3} \\
            T_f &= 814\si{\degree\kelvin} = 514\si{\celsius}
        \end{align*}
        Flash point of biodiesel \(\approx 130\)\si{\celsius}
    % -----------------------------------------------
    \vspace{0.1in}
    % -----------------------------------------------
    \item [1.41]
    \begin{enumerate}
        \item
            \begin{equation*}
                Q = mc \Delta T \quad c_{water} = 4.2 \frac{J}{g\si{\celsius}}
            \end{equation*}
            \begin{equation*}
                Q_{water} = (250)(4.2)(4) = 4200J
            \end{equation*}
        % ------------------------------------------
        \vspace{0.05in}
        % ------------------------------------------
        \item
            \begin{equation*}
                Q_{metal} = -Q_{water} = -4200J
            \end{equation*}
        % ------------------------------------------
        \vspace{0.05in}
        % ------------------------------------------
        \item
            \begin{align*}
                C_metal &= \frac{Q_{metal}}{\Delta T} \\
                &= \frac{-4200}{-4} \\
                &= 1050 \frac{J}{\si{\celsius}}
            \end{align*}
        % ------------------------------------------
        \vspace{0.05in}
        % ------------------------------------------
        \item
            \begin{align*}
                C_metal &= \frac{C_{metal}}{m} \\
                &= \frac{1050}{100} \\
                &= 10.5 \frac{J}{g \si{\celsius}}
            \end{align*}
    \end{enumerate}
    % -----------------------------------------------
    \vspace{0.1in}
    % -----------------------------------------------
    \item [1.54]
    \begin{enumerate}
        \item
            \begin{align*}
                W &= mgh \\
                &= (60)(9.81)(1500)
                &= 882900J
            \end{align*}
            \begin{equation*}
                100 kcal = 418400 J
            \end{equation*}
            \begin{equation*}
                \frac{882900}{418400 \times 0.25} = 8.441 oz = 239.3 g
            \end{equation*}
        % ------------------------------------------
        \vspace{0.05in}
        % ------------------------------------------
        \item
            \begin{equation*}
                Q = (8.441)(418400) \left( \frac{3}{4} \right) = 2648785.8 J
            \end{equation*}
            \begin{align*}
                Q &= mc \Delta T \\
                \Delta T &= \frac{Q}{mc_{water}} \\
                &= \frac{2678785.8}{(60000)(4.2)} \\
                \Delta T &= 10.5\si{\celsius}
            \end{align*}
        % ------------------------------------------
        \vspace{0.05in}
        % ------------------------------------------
        \item
            \begin{equation*}
                L = \frac{Q}{m} \quad m = \frac{Q}{L}
            \end{equation*}
            \begin{equation*}
                L = 580 \frac{cal}{g} = 2426.72 \frac{J}{g}
            \end{equation*}
            \begin{equation*}
                m = \frac{2648785.8}{2426.72} = 1092 g
            \end{equation*}
    \end{enumerate}
    % -----------------------------------------------
    \vspace{0.1in}
    % -----------------------------------------------
    \item [2.1]
    \begin{enumerate}
        \item
        Table 1: \\
            \begin{center}
                \begin{tabular}{ |c|c|c|c| }
                    \hline
                    Coin 1 & Coin 2 & Coin 3 & Coin 4 \\
                    \hline
                    H & H & H & H \\
                    T & T & T & T \\
                    H & T & T & T \\
                    T & H & T & T \\
                    T & T & H & T \\
                    T & T & T & H \\
                    H & H & T & T \\
                    T & H & H & T \\
                    T & T & H & H \\
                    H & T & T & H \\
                    T & H & T & H \\
                    H & T & H & T \\
                    H & H & H & T \\
                    H & H & T & H \\
                    H & T & H & H \\
                    T & H & H & H \\
                    \hline
                \end{tabular}
            \end{center}
        \item
            0 heads 1/16 \\
            1 heads 4/16 \\
            2 heads 6/16 \\
            3 heads 4/16 \\
            4 heads 1/16
    \end{enumerate}
% ---------------------------------------------------
%     End
% ---------------------------------------------------
\end{enumerate}
\end{document}