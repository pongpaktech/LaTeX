%%%%%%%%%%%%%%%%%%%%%%%%%%%%%%%%%%%%%%%%%
% Academic Title Page
% LaTeX Template
% Version 2.0 (17/7/17)
%
% This template was downloaded from:
% http://www.LaTeXTemplates.com
%
% Original author:
% WikiBooks (LaTeX - Title Creation) with modifications by:
% Vel (vel@latextemplates.com)
%
% License:
% CC BY-NC-SA 3.0 (http://creativecommons.org/licenses/by-nc-sa/3.0/)
% 
% Instructions for using this template:
% This title page is capable of being compiled as is. This is not useful for 
% including it in another document. To do this, you have two options: 
%
% 1) Copy/paste everything between \begin{document} and \end{document} 
% starting at \begin{titlepage} and paste this into another LaTeX file where you 
% want your title page.
% OR
% 2) Remove everything outside the \begin{titlepage} and \end{titlepage}, rename
% this file and move it to the same directory as the LaTeX file you wish to add it to. 
% Then add \input{./<new filename>.tex} to your LaTeX file where you want your
% title page.
%
%%%%%%%%%%%%%%%%%%%%%%%%%%%%%%%%%%%%%%%%%

%----------------------------------------------------------------------------------------
%	PACKAGES AND OTHER DOCUMENT CONFIGURATIONS
%----------------------------------------------------------------------------------------

\documentclass[11pt]{article}
\usepackage{geometry}
 \geometry{
 letterpaper,
 total={7.25 in, 9.375 in},
 left=0.625 in,
 top=0.875 in,
 headsep=14 pt,
 headheight=14 pt,
 }

\usepackage{fancyhdr}
 \pagestyle{fancy}
 \fancyhf{}
 \lhead{CTA Proposal}
 \rhead{SEGUE 1: An Unevolved Fossil Galaxy from the Early Universe}
 \rfoot{Techagumthorn \thepage}

\usepackage[utf8]{inputenc} % Required for inputting international characters
\usepackage[T1]{fontenc} % Output font encoding for international characters

\usepackage{mathpazo} % Palatino font
\usepackage{graphicx}
\usepackage{varioref}
\usepackage{float}
\usepackage{amssymb}


\begin{document}

%----------------------------------------------------------------------------------------
%	TITLE PAGE
%----------------------------------------------------------------------------------------

\begin{titlepage} % Suppresses displaying the page number on the title page and the subsequent page counts as page 1
	\newcommand{\HRule}{\rule{\linewidth}{0.5mm}} % Defines a new command for horizontal lines, change thickness here
	
	\center % Centre everything on the page
	
	%------------------------------------------------
	%	Headings
	%------------------------------------------------
	
	\textsc{\LARGE University of Washington, Bothell}\\[1.5cm] % Main heading such as the name of your university/college
	
	\textsc{\Large CTA Proposal}\\[0.5cm] % Major heading such as project name
	
	\textsc{\large B Phys 311, Intro to Astrophysics}\\[0.5cm] % Minor heading such as course title
	
	%------------------------------------------------
	%	Title
	%------------------------------------------------
	
	\HRule\\[0.4cm]
	
	{\huge\bfseries SEGUE 1: An Unevolved Fossil Galaxy from the Early Universe}\\[0.4cm] % Title of your document
	
	\HRule\\[1.5cm]
	
	%------------------------------------------------
	%	Author(s)
	%------------------------------------------------
	
	\begin{minipage}{0.4\textwidth}
		\begin{flushleft}
			\large
			\textit{Pongpak Techagumthorn}\\
			\textsc{pongpakt@uw.edu} % Your name
		\end{flushleft}
	\end{minipage}
	
	%------------------------------------------------
	%	Date
	%------------------------------------------------
	
	\vfill\vfill\vfill % Position the date 3/4 down the remaining page
	
	{\large January 25, 2021} % Date, change the \today to a set date if you want to be precise
	
	\vfill % Push the date up 1/4 of the remaining page
	
\end{titlepage}

%----------------------------------------------------------------------------------------

\section{Topic Background}
When studying the early universe scientists commonly seek out stars with low metal content. Locked inside these stars
is information about the physical and chemical conditions that led to the formation of the star. The low metallicity of
these stars, specifically low $\left[ Fe / H \right]$ values and high $\left[ \alpha / Fe \right]$ values, indicate that
the stars were formed in the early universe. By comparing the quantity of $\alpha$-elements, iron, and hydrogen we can map the
timescales of nucleosynthesis to the timescales of different types of supernovae. The $\alpha$-elements are carbon, oxygen, neon,
magnesium, silicon, sulfur, argon, and calcium.

The earliest supernovae events in the universe were core-collapse supernovae of Massive stars. Massive stars produce large
amounts of $\alpha$-elements during their stellar evolution and core-collapse explosion. Massive stars have short lifetimes
of less than 10 million years. Heavy metals, such as iron, appeared later in the astronomical timeline as a byproduct of Type Ia
supernovae. Type Ia supernovae originate from binary star systems containing a white dwarf stars. These binary systems have much
longer lifetimes, on the order of $10^8$ years, before collapsing into a supernova. These substantially different timelines allow
scientists to approximately date the relative timing of nucleosynthesis in early stars.

For decades scientists looked at low metal stars in the halo of the Milky Way to study the early universe. In the past decade efforts
such as the Sloan Digital Sky Survey (SDSS) have uncovered a plethora of very dim dwarf spheroidal (dSph) galaxies and ultra faint dwarfs
with total luminosities in the range of \(10^{5} L_{\odot} \lesssim L \lesssim 10^{7} L_{\odot}\) and \(L \lesssim 10^5 L_{\odot}\),
respectively. Within this pool of dSph and ultra faint dwarf galaxies scientists discovered a small handful of ultra faint dwarf galaxies
that showed extremely low $\left[ Fe / H \right]$ values. In this small handful one galaxy, SEGUE 1 stood out from the rest.
SEGUE 1 stood out since it's stars show increasing metallicity but do not exhibit decreasing $\left[\alpha / Fe\right]$ ratios.
This would indicate that SEGUE 1 dates back to the very early universe predating Type Ia supernovae events.

\section{Novelty}

The authors of this paper made new spectrography measurements of 6 stars within SEGUE 1 focusing on the chemical enrichment process
to better understand star formation in the very early universe. 

\section{Results, Discussion, and Future Work}

\clearpage

\section{References}

\end{document}

